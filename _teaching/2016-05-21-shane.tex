---
title: "Dark Matter "
collection: teaching
teaching_type: "Workshop series"
permalink: /teaching/2016-05-21-shane
institution: "Lick Observatory"
date: 2016-05-21
excerpt: "In this experiment we measure the galactic rotation curve of the edge-on spiral galaxy UGC 08787.  We derive a theoretical model of the rotation curve by calculating its mass from a measured luminosity curve in order to compare to the measured rotation curve of UGC 08787 to conclude if dark matter is present. "
---
We take a spectroscopic survey of the UGC 08787 edge-on Sbc galaxy centered around H?, with cosmological redshift z = 0.01535 � 0.0002053.  We correct the spectra With the Lick Observatory KAST double spectrograph with a Neon lamp spectra and line fit ?= 0.063x + 577.31 to convert pixel position to wavelength.  We measure the rotation curve of the galaxy using theDoppler effect of the spiral arms of UGC 08787 yielding redshifted and blue-shifted stars in the galaxy. From a theoretical rotation curve model derived from the luminous matter of UGC 08787 we determine that there is dark matter in the galaxy that is raising the rotational velocity of the stars in the galaxy.
